\documentclass[a4paper, 12pt]{report}

\usepackage{graphicx}
\usepackage{xcolor}
\usepackage{listings}
\usepackage[T1]{fontenc}
\usepackage[utf8]{inputenc}
\usepackage[polish]{babel}
\usepackage{caption}
\usepackage{hyperref}


% image folder path
\graphicspath{{./pix/}}

% change tabel of contents label
\renewcommand\contentsname{Spis treści}

\setcounter{tocdepth}{4}
\setcounter{secnumdepth}{4}

% listings configuration
\lstset{language=C++,
	basicstyle=\scriptsize,
	breakatwhitespace=false,
	breaklines=true,
	commentstyle=\color{green},
	frame=single,
	keepspaces=false,
	keywordstyle=\color{blue},
	numbers=left,
	numbersep=5pt,
	showspaces=false,
	showtabs=false,
	stringstyle=\color{red},
	tabsize=2,
}


% document's beginning
\begin{document}
\begin{titlepage}
\begin{center}
		\vspace*{1cm}
		\textbf{Grafika komputerowa\\Laboratorium\\}
		\vspace{2cm}
		\textbf{Stanislau Antanovich \& Mykola Sharonov}
		\vfill
		\vspace{0.8cm}
		\includegraphics[scale=0.7]{logo.png}
\end{center}
\end{titlepage}

\tableofcontents
\newpage

\chapter{Budowa obiektu sterowanego}
\section{Opis zadania}
Należy zbudować ``robot rolniczy (łazik)'' wykorzystując wyłącznie prymitywy bazujące na trójkącie. Obiekt ten będzie wykorzystywany na kolejnych zajęciach. W tworzonej grze komputerowej użytkownik będzie miał możliwość sterowania tym łazikiem.
\section{Wymagania}
Wymagania dotyczące budowy głósnego obiektu:
\begin{itemize}
\item Na ocenę 3: Obiekt złożony z co najmniej 10 brył elementarnych (walec, prostopadłościan, itp.) zbudowanych przy użyciu prymitywów bazujących na trójkącie.
\item Na ocenę 4: Obiekt złożony z co najmniej 20 brył elementarnych (walec, prostopadłościan, itp.) zbudowanych przy użyciu prymitywów bazujących na trójkącie.
\item Na ocenę 5: Obiekt złożony z co najmniej 25 brył elementarnych (walec, prostopadłościan, itp.) zbudowanych przy użyciu prymitywów bazujących na trójkącie oraz projekt napisany obiektowo w C++.

Możliwość zaimportowania łazika z programu graficznego (np. Blender) o budowie odpowiadającej co najmniej 25 bryłom elementarnym.
\end{itemize}

\section{Realizaja zadania}

Naszym ``łazikiem'' będzie występował zwykły samochód.

\begin{figure}[h]
		\centering
		\includegraphics{lazik.jpg}
		\caption{Łazik(samoshód)}
\end{figure}


\subsection{class \emph{Wheel}}
\subsubsection{Opis działania}

Klasa \emph{Wheel} odpowiada za rysowanie koła.

\subsubsection{Plik \emph{Wheel.h}}

Plik \emph{Wheel.h} deklaruje wszystkie zmienne oraz metody, które będą używane obiektami tej klasy.

\lstinputlisting[firstline=2]{../WindowsProject1/Wheel.h}

\subsubsection{Plik \emph{Wheel.cpp}}

Plik \emph{Wheel.cpp} zawiera inicjalizacje zmiennych oraz metod opisanych w pliku nagłówkowym \emph{Wheel.h}.

\lstinputlisting{../WindowsProject1/Wheel.cpp}

\subsection{class \emph{SideSciana}}
\subsubsection{Opis działania}

Klasa \emph{SideSciana} odpowiada za rysowanie scian bokowych samochodu.

\subsubsection{Plik \emph{SideSciana.h}}

Plik nagłówkowy \emph{SideSciana.h} deklaruje wszystkie zmienne oraz metody, które będą używane obiektami tej klasy.

\lstinputlisting[firstline=2]{../WindowsProject1/SideSciana.h}

\subsubsection{Plik \emph{SideSciana.cpp}}

Plik \emph{SideSciana.cpp} zawiera inicjalizacje zmiennych oraz metod opisanych w pliku nagłówkowym \emph{SideSciana.h}.

\lstinputlisting{../WindowsProject1/SideSciana.cpp}

\subsection{class \emph{Front}}
\subsubsection{Opis działania}
Klasa \emph{Front} odpowiada za rysowanie przdniej ściany samochodu.

\subsubsection{Plik \emph{Front.h}}
Plik nagłówkowy \emph{Front.h} deklaruje wszystkie zmienne oraz metody, które będą używane obiektami tej klasy.
\lstinputlisting[firstline=2]{../WindowsProject1/Front.h}

\subsubsection{Plik \emph{Front.cpp}}
Plik \emph{Front.cpp} zawiera inicjalizacje zmiennych oraz metod opisanych w pliku nagłówkowym \emph{Front.h}.
\lstinputlisting{../WindowsProject1/Front.cpp}

\subsection{class \emph{Back}}
\subsubsection{Opis działania}
Klasa \emph{Back} odpowiada za rysowanie tylnej ściany samochodu.

\subsubsection{Plik \emph{Back.h}}
Plik nagłówkowy \emph{Back.h} deklaruje wszystkie zmienne oraz metody, które będą używane obiektami tej klasy.
\lstinputlisting[firstline=2]{../WindowsProject1/Back.h}

\subsubsection{Plik \emph{Back.cpp}}
Plik \emph{Back.cpp} zawiera inicjalizacje zmiennych oraz metod opisanych w pliku nagłówkowym \emph{Back.h}.
\lstinputlisting{../WindowsProject1/Back.cpp}

\chapter{Budowa otoczenia}
\section{Opis zadania}
Należy zbudować elementy otoczenia, w którym będzie poruszał się robot rolniczy wykorzystując wyłącznie prymitywy bazujące na trójkącie. Elementy te będą wykorzystywane na kolejnych zajęciach i będą powiązanie z fabułą gry.
\section{Wymagania}
Wymagania dotyczące budowy otoczenia:
\begin{itemize}
\item Na ocenę 3: Przygotowanie otoczenia o podłożu płaskim oraz utworzenie dwóch obiektów dodatkowych (drzewo, bramka, budynek).
\item Na ocenę 4: Przygotowanie otoczenia o podłożu nieregularnym (góra, stadion, wyboista ziemia) oraz utworzenie jednego obiektu dodatkowego.
\item Na ocenę 5: Import otoczenia z programu graficznego (otoczenie o podłożu nieregularnym i minimum 1 obiekt dodatkowy).
\end{itemize}

\section{Realizaja zadania}

\begin{figure}
		\centering
		\includegraphics[scale=0.4]{podloze.jpg}
		\caption{Podłoże}
\end{figure}

\subsection{class Podloze}

Klasa \emph{Podloze} odpowiada za rysowanie otoczenia. Plik \emph{Podloze.cpp} zawiera ponat 29 tys. linii kodu więc nie będzie umieszczony w sprawozdaniu.

\subsubsection{Opis działania}

Podłoże zostało eksportowane z programu \emph{Blender}.

\subsubsection{Plik \emph{Podloze.h}}

\lstinputlisting{../WindowsProject1/Podloze.h}

\chapter{Teksturowanie}
\section{Opis zadania}

Należy dokonać teksturowania według przedstawionych poniżej kryteriów.
\section{Wymagania}

Wymagania dotyczące dodania teksurowania.
\begin{itemize}
\item Na ocenę 3: Teksturowanie obiektów otoczenia oraz utworzenie autorskiego rozwiązania sterowaniem kamerą.
\item Na ocenę 4: Jak na ocenę 3 oraz teksturowanie powierzchni.
\item Na ocenę 5: Jak na ocenę 4 oraz teksturowanie obiektu, który będzie sterowany (minimum 3 bryły).
\end{itemize}
\section{Realizacja zadania}

\lstinputlisting[firstline=521, lastline=531]{../WindowsProject1/main.cpp}

\chapter{Sterowanie obiektem głównym}
\section{Opis zadania}
Należy dokonać sterowanie obiektem głównym.
\section{Wymagania}
Wymagania dotyczące sterowania obiektem głównym.
\begin{itemize}
\item Na ocenę 3: Realizacja prostego sterowanie przód-tył i obrót wokół własnej osi.
\item Na ocenę 4: Implementacja prostej fizyki sterowania (w przypadku łazika różnica prędkości na gąsienicach lub oś skrętna).
\item Na ocenę 5: Jak na ocenę 4 oraz implementacja podstawowych zagadnień fizycznych np. pęd ciała.
\end{itemize}
\section{Realizacja zadania}

W pliku głównym \emph{main.cpp} są 8 zmiennych odpowiadających za naciśnięcie klawisz.

\lstinputlisting[firstline=27,lastline=34]{../WindowsProject1/MAIN.cpp}

Niżej w tym samym pliku znajduje się \verb|switch\case|, który cały czas sprawdza czy klawisze są naciśnięte i w zależności od tego zmienne przyjmują inne wartości(\verb|true| albo \verb|false|).

\lstinputlisting[firstline=897,lastline=970]{../WindowsProject1/MAIN.cpp}

W funkcji \emph{RenderScene} znajduje się fragment kodu, który sprawdza stany zmiennych.

\lstinputlisting[firstline=532, lastline=547]{../WindowsProject1/MAIN.cpp}

W zależności od wartości zmiennej będą również zmienione zmienne: \verb|posX|, \verb|posY|, \verb|posZ|. Wartości odpowiadają za pozycję samochodui na płaszczyźnie. 

\end{document}
